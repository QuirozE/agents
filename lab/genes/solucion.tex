\documentclass[twocolumn]{article}

\usepackage[margin=1.5cm]{geometry}

\usepackage[spanish]{babel}

\usepackage{arev}

\usepackage{hyperref}
\usepackage[spanish]{cleveref}

\title{
  Modelación y simulación computacional basada en agentes\\
  Práctica 3: Algoritmos evolutivos
}

\author{
  Edgar Quiroz
}

\begin{document}
\maketitle

\begin{abstract}
  Se presenta implementación de modelos de algoritmos genéticos y programación
  evolutiva para resolver problemas de otpmización de funciones continuas, de
  combinatoria y de juegos.
\end{abstract}

\section{Algoritmo Genético Simple}

Implementar el esquema vist en clase del Algoritmo Genético Simple para
maximizar funciones de dos variables considerando los siguientes puntos

\begin{itemize}
  \item Representación de los individuos: cadenas de bits
  \item Inicialización de la población: a través de un generador de números
    pseudo-aleatorios.
  \item Tamaño de la población: constante
  \item Condición de término: Dos opciones
    \begin{itemize}
      \item Número fijo de generaciones
      \item Cuando ya no haya mejoría de los individuos
    \end{itemize}
  \item Mecanismo de selección: Proporicional a su aptitud (ruleta)
  \item Recombinación: 1-crossover
  \item Mutación: bitflip, misma probabilidad para todo el genotipo
  \item Operador de reemplazo: generacional, es decir, se renueva toda la
    población en la siguiente generación
\end{itemize}

Para encontrar el \textbf{máximo global} de la \textbf{función de Schwefel}
(\cref{eq:schwfel}) para $d = 2$ en el intervalo $(-500 \times 500)^{2}$. Las
soluciones deberán tener al menos seis decimalaies de precisión.

\begin{equation}\label{eq:schwfel}
  f(x_{1}, \dots, x_{d}) = \sum_{i = 1}^{d}{x_{i}\sin{(\sqrt{|x_{i}|})}}
\end{equation}

\subsection{Modelación}

\subsection{Simulación}

Establezca una población fija de 100 individuos, con tasa de recombinación en
$pc = 0.5$, tasa de mutación $pm = 0.1$ y condición de término al finalizar 500
generaciones.

\begin{itemize}
  \item ¿Cuál fue el mejor individuo? Reporte la estrcutura del genotipo,
    fenotipo y fitness. ¿Es el máximo global?
  \item Grafique la evolución de la mejor aptitud, aptitud promedio y fitness
    promedio.
  \item Graique en el plano $x, y$ los individuos de las generaciones 1, 10,
    50, 100.
\end{itemize}

\subsection{Exploración}

Genere otros experimentos cambiando la tasa de recombinacioń y de mutación.
Grafique los puntos anteriores para las diferentes configuraciones. ¿Con qué
parámetros se encuentra más rápido el máximo global?

\section{Las ocho reinas}

El problema de las ocho reinas consiste en colocar ocho reinas en un tablero de
ajedrez si que se puedan capturar entre sí en un movimiento. El problema fue
propuesto en 1848 por Max Bezzel. Tiene 92 posibles soluciones, contando
reflexiones y rotaciones.

Considere una solución con las siguientes características

\begin{itemize}
  \item Representación: Permutación
  \item Recombinación: \texttt{cut \& crossfit} con probabilidad 1
  \item Mutación: \texttt{swap} con probabilidad 0.8
  \item Selección de padres: mejores 2 de 5 aleatorios
  \item Operador de reemplazo: reemplazar a los peores
  \item Tamaño de la población: 100 individuos
  \item Inicialización: Aleatoria
  \item Condición de paro: Encontrar la solución o evaluación de 10,000
    fitness.
\end{itemize}

A partir del planteamiento realizado en clase, resolver este problema a través
de algoritmos evolutivos.
\begin{itemize}
  \item Reporte la solución: genotipo, fenotipo y fitness.
  \item Grafique la evolución de la mejor aptitud y fitness promedio.
  \item Con su implementación resuelva el problema para $n=30$ reinas.
\end{itemize}

\section{Hormiga artificial}

En 1991, David Jefferson y Robert Collins desarrollaron un mecanismo para
``adaptar'' un organismo artificail a un ambiente dado. El objetivo es conducir
una hormiga artificial a través de un camino irregular formado por rastros de
comida.

La hormiga artificial se desarrolla en una retícula toroide de $32\times 32$
sobre el plano. La hormiga inicia su recorrido en la celda superior izquierda
identificada con las coordenadas $(0, 0)$ viendo hacia el este. El sistema de
coordenadas es $(r, c)$.

El recorrido que intentará la hormiga se denomina ``Santa Fe Trail'', que
consiste en 89 migajas de comida distribuidas sobre la retícula como se muestra
en la figura $k$. Las celdas en negro con las migajas de comida y las celdas en
amarillo son huecos donde se pierde el rastro. Exist huecos únicos, huecos
dobles, huecos únicos en las orillas y huecos dobles en las orillas. Este
rastros fue diseñado por Christopher Langton. Los números indican el conteo de
las 89 migaas de comida.

La hormiga artificial tiene una visión limitada del mundo. En particular la
horimga tiene un sensor con el cual solo puede ver lo que hau inmediatamente en
la celda adyacente en la dirección que está viendo.

La hormiga puede ejecutar cualquiera de las siguientes acciones

\begin{itemize}
  \item \textsc{RIGTH}: gira la hormiga $90^{\circ}$ a la derecha, mateniendo la
    posición
  \item \textsc{LEFT}: gira la hormiga $90^{\circ}$ a la izquierda, manteniendo
    la posición.
  \item \textsc{MOVE}: mueve la hormiga hacia adelante en la dirección que está
    ``viendo''. Cuando se mueve a la ceda sguiente se come la comida, eliminando
    el rastro de comida.
\end{itemize}

La hormaiga también cuenta con las siguiente funciones

\begin{itemize}
  \item \textsc{IF-FOOD-AHEAD}: es un operador condicional de 2 argumentos.
    Ejecuta una acción dependiendo si hay o no comida.
  \item \textsc{PROGN2}: operador no condicional. Ejecuta dos instrucciones de
    forma secuencial.
  \item \textsc{PROGN3}: idem \textsc{PROG2}, pero con tres instrucciones.
\end{itemize}

El problema queda determinado por los siguientes puntos

\begin{itemize}
  \item Objetivo: Encontrar un programa que controle el funcionamiento de la
    hormiga artificial tal que encuentre las 89 migajas de comida en el rastro.
  \item Conjunto terminal: \textsc{LEFT, RIGHT,MOVE}
  \item Conjunto función: \textsc{IF-FOOD-AHEAD, PROGN2, PROGN3}
  \item Casos de fitness: un solo caso
  \item Raw fitness: número de piezas de comida levantadas antes de 400
    operaciones.
  \item Parámetros: población de tamaño 500 y 51 generaciones.
\end{itemize}

Usando programación genétca, encuentra la mejor estrategia para que la hormiga
artificial se coma todas lasmigajas en un tiempo dada.

\begin{itemize}
  \item Encontrar el programa más adecuado para resolver el problema.
  \item Graficar al evolución del ``fitness'' del mejor individuo y
    ``fitness'' promedio.
\end{itemize}

\end{document}